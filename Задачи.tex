\documentclass[a4paper,10pt]{article}
\usepackage[utf8]{inputenc}
\usepackage[russian]{babel}
\usepackage{listings}
\usepackage{fullpage}

\begin{document}
\lstset{language=Haskell} 
\setcounter{secnumdepth}{0}
\title{Задачи по Haskell}
\date{}
\maketitle
При решении задач можно использовать сопостовление по шаблону и свои имена переменных у функций.
При решении задач можно использовать любые функции из модуля Prelude, а использование любых других
модулей запрещено, если в условии задачи не сказано иное.
\section{Задачи на списки}
\begin{enumerate}

\item {
\begin{lstlisting}
mul :: Num a => [[a]] -> [[a]] -> [[a]] 
mul a b = ?
\end{lstlisting}
Перемножение двух матриц. Предполагается, что у матриц подходящие размеры. Можно использовать
функцию transpose из модуля Data.List, подключать модуль не нужно.
\begin{lstlisting}
mul [[1,2],[3,4]] [[0],[1]] = [[2],[4]]
\end{lstlisting}
}

\item {
\begin{lstlisting}
divide :: (a -> Bool) -> [a] -> [[a]]
divide p xs = ?
\end{lstlisting}
Написать функцию divide, которая принимает предикат p и список xs и разбивает xs на список
списков, начиная новый список каждый раз, когда изменяется значение предиката.
\begin{lstlisting}
divide even [] = []
divide even [1,2,3,4] = [[1],[2],[3],[4]]
divide even [1,3,5,4,2] = [[1,3,5],[4,2]]
divide even [3,5,7] = [[3,5,7]]
\end{lstlisting}
}

\item {
\begin{lstlisting}
compress :: [a] -> [a]
compress xs = ?
\end{lstlisting}
Если в списке идут подряд одинаковые элементы, то нужно оставить из них только одно значение.
\begin{lstlisting}
compress "aaaabccaadeeee" = "abcade"
\end{lstlisting}
}

\item {
\begin{lstlisting}
encode :: [a] -> [(Int,a)]
encode xs = ?
\end{lstlisting}
Результирующий список состоит из пар (число вхождений подряд элемента, элемент).
\begin{lstlisting}
encode "aaabccaadeee" = [(3,'a'),(1,'b'),(2,'c'),(2,'a'),(1,'d'),(3,'e')]
\end{lstlisting}
}

\item {
\begin{lstlisting}
repli :: [a] -> Int -> [a]
repli xs n = ?
\end{lstlisting}
Каждый элемент в списке xs повторить n раз.
\begin{lstlisting}
repli "abc" 3 = "aaabbbccc"
\end{lstlisting}
}

\item {
\begin{lstlisting}
change :: a -> [[a]]
change n = ?
\end{lstlisting}
Пусть есть список положительных достоинств монет coins, отсортированный по возрастанию.
Напишите функцию change, которая разбивает переданную ей положительную сумму денег на монеты
достоинств из списка coins всеми возможными способами.
Порядок монет в каждом разбиении имеет значение, то есть наборы [2,2,3] и [2,3,2] — различаются.
Список coins определять не надо.
Например, если coins = [2, 3, 7]
\begin{lstlisting}
change 7 = [[2,2,3],[2,3,2],[3,2,2],[7]]
\end{lstlisting}
}

\item {
\begin{lstlisting}
seqA :: Int -> Integer
seqA n = ?
\end{lstlisting}
Функция seqA должна находить элементы следующей рекуррентной последовательности:
$$a_0 = 1; a_1 = 2; a_2 = 3; a_{k+3} = a_{k+2} + a_{k+1} - 2a_k$$
\begin{lstlisting}
seqA 0 = 1
seqA 4 = 2
seqA 5 = -1
\end{lstlisting}
}

\end{enumerate}
\end{document}

