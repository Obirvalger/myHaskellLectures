\documentclass[a4paper,10pt]{article}
\usepackage[utf8]{inputenc}
\usepackage[russian]{babel}
\usepackage{listings}
\usepackage{fullpage}

\begin{document}
\lstset{language=Haskell} 
\setcounter{secnumdepth}{0}
\title{План лекций}
\date{}
\maketitle

\section{Списки}
\subsection{Введение}
Компилятор ghc, ghci, Haskell Platform.

Haskell – чисто функциональный, типизированный язык программирования.

Чистые функции.

Типы Int, Integrer, Float, Double, Bool = True | False, Char.

Арифметические операции.
\begin{lstlisting}
+, -, *, /, div, mod
\end{lstlisting}

Тип функции:
\begin{lstlisting}
not :: Bool -> Bool
not False = True
not True  = False

plus :: Int -> Int -> Int
plus x y = x + y

plus3 :: Int -> Int
plus3 = plus 3
\end{lstlisting}

Кортежи (a,b).
\begin{lstlisting}
fst (x,y) = x

snd (x,y) = y

('a',True)
\end{lstlisting}

Списки
\begin{lstlisting}
[a] = [] | a : [a]
[]
1:2:[]
[1,2]
[1..3] = [1,2,3]
[0,2..8] = [0,2,4,6,8]
[1,1.5..3] = [1.0,1.5,2.0,2.5,3.0]
\end{lstlisting}

Конструктор списков (list comprehensions)
\begin{lstlisting}
[x | x <- [1..3]] = [1,2,3]
[(x,y) | x <- [1,2], y <- [1,2]] = [(1,1), (1,2), (2,1), (2,2)]
[(x,y) | x <- [1..3], y <- [1..4], x == y] = [(1,1), (2,2), (3,3)]
\end{lstlisting}

\clearpage

\subsection{Базовые функции со списками}
\begin{lstlisting}
head :: [a] -> a
head (x:xs) = x

tail :: [a] -> [a]
tail (x:xs) = xs

length           :: [a] -> Int
length []        =  0
length (x:xs)     =  1 + length xs

(++) :: [a] -> [a] -> [a]
(++) [] ys     = ys
(++) (x:xs) ys = x : (xs ++ ys)

(!!) :: [a] -> Int -> a
(x:_)  !! 0 = x
(_:xs) !! n = xs !! (n-1)

reverse :: [a] -> [a]
reverse [] = []
reverse (x:xs) = reverse xs ++ [x]

reverse l =  rev l [] where
    rev []     a = a
    rev (x:xs) a = rev xs (x:a)

[1,2,3] []
[2,3] 1:[]
[3] 2:1:[]
[] 3:2:1:[]
[] [3,2,1]

take :: Int -> [a] -> [a]
take _ []     = []
take n (x:xs) | n <= 0    = []
              | otherwise = x : take (n-1) xs

drop

\end{lstlisting}

\subsection{Бесконечные списки}
\begin{lstlisting}
[1..]

[2,4..]

take 5 [1..]
[1,2,3,4,5]

repeat :: a -> [a]
repeat x = x : repeat x

take 2 (repeat 3)
[3,3]

take 2 (3 : repeat 3)
3 : take 1 (repeat 3)
3 : take 1 (3 : repeat 3)
3 : 3 : take 0 (repeat 3)
3 : 3 : take 0 (3 : repeat 3)
3 : 3 : [] = [3,3]


($) :: (a -> b) -> a -> b
f $ x = f x

replicate :: Int -> a -> [a]
replicate n x = take n $ repeat x

cycle :: [a] -> [a]
cycle xs = xs ++ cycle xs

take 5 $ cycle [1,2]
[1,2,1,2,1]

iterate :: (a -> a) -> a -> [a]
iterate f x = x : iterate f (f x)
\end{lstlisting}

Линейный генератор 
\begin{lstlisting}
f x = mod (5*x + 3) 11
take 5 $ iterate f 1
[1,8,10,9,4]
\end{lstlisting}

\subsection{Функции высших порядков}

\begin{lstlisting}
takeWhile :: (a -> Bool) -> [a] -> [a]
takeWhile _ []     = []
takeWhile p (x:xs) | p x       = x : takeWhile p xs
                   | otherwise = []

dropWhile

filter :: (a -> Bool) -> [a] -> [a]
filter _ [] = []
filter p (x:xs) = if p x then x : filter xs else filter xs

filter even [1..5]
[2,4]

filter (not . even) [1..5]
[1,3,5]

(.) :: (b -> c) -> (a -> b) -> a -> c
f . g = \x -> f (g x)
\end{lstlisting}

Решето Эратосфена

\begin{lstlisting}
sieve :: [Integrer] -> [Integrer]
sieve (x:xs) = x : sieve (filter (\y -> y `mod` x /= 0) xs)

primes = sieve [2..]
\end{lstlisting}

\clearpage

\subsubsection{Map и zipWith}

\begin{lstlisting}
map :: (a -> b) -> a -> b
map f []     = []
map f (x:xs) = f x : map f xs

map (^2) [1..5]
[1,4,9,16,25]

map (2^) [1..5]
[2,4,8,16,32]

zipWith :: (a -> b -> c) -> [a] -> [b] -> [c]
zipWith f (x:xs) (y:ys) = f x y : zipWith f xs ys
zipWith _ _ _           = []

zipWith3 :: (a -> b -> c -> d) -> [a] -> [b] -> [c] -> [d]

fibs = 0:1:zipWith (+) fibs (tail fibs)
\end{lstlisting}
Вычисление в стратегии сверху вниз:
\begin{lstlisting}
fib n = fibs !! n

fib 3
2

fibs !! 3
(0:1:zipWith (+) fibs (tail fibs)) !! 3
(1:zipWith (+) fibs (tail fibs)) !! 2
(zipWith (+) fibs (tail fibs)) !! 1
(0 + 1 : zipWith (+)
    (1:zipWith (+) fibs (tail fibs))
    (zipWith (+) fibs (tail fibs))) !! 1
(zipWith (+)
    (1:zipWith (+) fibs (tail fibs))
    (zipWith (+) fibs (tail fibs))) !! 0
(zipWith (+)
    (1:zipWith (+) fibs (tail fibs))
    (zipWith (+)
        (0:1:zipWith (+) fibs (tail fibs))
        (1:zipWith (+) fibs (tail fibs)))) !! 0
(zipWith (+)
    (1:zipWith (+) fibs (tail fibs))
    (0 + 1 : zipWith (+)
        (1:zipWith (+) fibs (tail fibs))
        (zipWith (+) fibs (tail fibs)))) !! 0
(1 + 1 : zipWith (+)
    (zipWith (+) fibs (tail fibs))
    (zipWith (+)
        (1:zipWith (+) fibs (tail fibs))
        (zipWith (+) fibs (tail fibs)))) !! 0
2
\end{lstlisting}

В Haskell применяется измененная версия этой стратегии.
Создается ссылка на fibs и fibs будет вычисляться только один раз, на каждом шаге добавляя новые
элементы.

\clearpage

\subsection{Свёртка}

\begin{lstlisting}
sum []     = 0
sum (x:xs) = x + sum xs

concat []       = []
concat (xs:xss) = xs ++ concat xss

foldr :: (a -> b -> b) -> b -> [a] -> b
foldr f e []     = e
foldr f e (x:xs) = f x foldr f e xs

sum = foldr (+) 0
concat = foldr (++) []

foldl :: (b -> a -> b) -> b -> [a] -> b
foldl f e []     = e
foldl f e (x:xs) = foldl f (f e x) xs

reverse = foldl (flip (:)) []

flip :: (a -> b -> c) -> b -> a -> c
flip f x y = f y x

foldr1 :: (a -> a -> a) -> [a] -> a
foldr1 f [x]    = x
foldr1 f (x:xs) = f x foldr1 f xs

maximum = foldr1 max

filter p = foldr (\x xs -> if p x then x:xs else xs) []
map f = foldr ((:) . f) []
length = foldr (\_ n -> 1 + n) 0

\end{lstlisting}

\subsection{Data.List и сортировки}
\begin{lstlisting}
transpose []             = []
transpose ([]   : xss)   = transpose xss
transpose ((x:xs) : xss) =
    (x : [h | (h:_) <- xss]) : transpose (xs : [ t | (_:t) <- xss])

qsort :: Ord a => [a] -> [a]
qsort []     = []
qsort (x:xs) = qsort (filter (<=x) xs) ++ [x] ++ qsort (filter (>x) xs)

isort :: Ord a => [a] -> [a]
isort []     = []
isort (x:xs) = insert x (isort xs) where
  insert x []     = [x]
  insert x ys@(y:ys') | x > y     = y : insert x ys'
                      | otherwise = x : ys
\end{lstlisting}

\clearpage

\begin{lstlisting}
msort :: Ord a => [a] -> [a]
msort = mergeAll . sequences
  where
    sequences (a:b:xs)
      | a > b     = descending b [a]  xs
      | otherwise = ascending  b (a:) xs
    sequences xs  = [xs]

    descending a as bs@(b:bs')
      | a > b          = descending b (a:as) bs'
    descending a as bs = (a:as): sequences bs

    ascending a as bs@(b:bs')
      | a <= b        = ascending b (\ys -> as (a:ys)) bs'
    ascending a as bs = as [a]: sequences bs

    mergeAll [x] = x
    mergeAll xs  = mergeAll (mergePairs xs)

    mergePairs (a:b:xs) = merge a b: mergePairs xs
    mergePairs xs       = xs

    merge as@(a:as') bs@(b:bs')
      | a > b     = b:merge as  bs'
      | otherwise = a:merge as' bs
    merge [] bs   = bs
    merge as []   = as
\end{lstlisting}

\section{Монады}
\subsection{Maybe}
\begin{lstlisting}
data Maybe a = Nothing | Just a

safeHead :: [a] -> Maybe a
safeHead []    = Nothing
safeHead (x:_) = Just x

lookup                  :: (Eq a) => a -> [(a,b)] -> Maybe b
lookup _ []             =  Nothing
lookup key ((k,v):ps)
    | key == k          =  Just v
    | otherwise         =  lookup key ps

lookup 2 [(1,"one"), (2,"two"), (3,"three")]
Just "two"

lookup 4 [(1,"one"), (2,"two"), (3,"three")]
Nothing

class Monad m where
    return :: a -> m a
    (>>=)  :: m a -> (a -> m b) -> m b
\end{lstlisting}

\clearpage

\begin{lstlisting}
instance Monad Maybe where
    return x = Just x
    Nothing  >>= f = Nothing
    (Just x) >>= f = f x
\end{lstlisting}

Последовательный поиск в нескольких списках (аналогичный подход при поиске в БД).
\begin{lstlisting}
empDep          = [("Mike", "It"), ("Jan", "Sales")]
depCountry      = [("It", "Japan"), ("Sales", "USA")]
countryCurrency = [("Japan", "JPY"), ("USA", "USD")]
currencyRate    = [("JPY", 112), ("USD", 1)]

f :: String -> Maybe Int
f emp = case lookup emp empDep of
          Nothing   -> Nothing
          Just dep  -> case lookup dep depCountry of
                         Nothing      -> Nothing
                         Just country -> case lookup country countryCurrency of
                                           Nothing       -> Nothing
                                           Just curr -> lookup curr currencyRate

fB emp = lookup' empDep emp >>= lookup' depCountry >>= lookup' countryCurrency
         >>= lookup' currencyRate where
           lookup' ps k = lookup k ps

fD emp = do dep <- lookup emp empDep
            country <- lookup dep depCountry
            currency <- lookup country countryCurrency
            lookup currency currencyRate
\end{lstlisting}

\subsection{List}
\begin{lstlisting}
instance Monad [a] where
    return x = [x]
    xs >>= f = concat (map f xs)
\end{lstlisting}

Построение грамматик.
\begin{lstlisting}
f 'a' = "ca"
f 'b' = "bb"
f 'c' = "a"

take 3 $ iterate (>>= f) "abc"
["abc","cabba","acabbbbca"]
\end{lstlisting}

\subsection{IO}
\begin{lstlisting}
class Monad m where
    return :: a -> m a
    (>>=)  :: m a -> (a -> m b) -> m b
    (>>)   :: m a -> m b -> m b
    m >> k = m >>= \x -> k

putChar :: Char -> IO ()

putStrLn :: [Char] -> IO ()
putStrLn xs = foldr (>>) (return ()) (map putChar xs) >> putChar '\n'

getChar :: IO Char

getLine :: IO [Char]
getLine = do x <- getChar
             if x == '\n'
             then return ""
             else do xs <- getLine
                     return (x:xs)
\end{lstlisting}

Структура программы.

Создайте файл hello.hs, со следующим содержимым и скомпелируйте: {\tt ghc hello.hs}.
\begin{lstlisting}
module Main where

main :: IO ()
main = putStrLn "Hello, World!"
\end{lstlisting}
\end{document}
