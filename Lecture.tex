\documentclass[a4paper,10pt]{article}
\usepackage[utf8]{inputenc}
\usepackage[russian]{babel}
\usepackage{listings}
\usepackage{fullpage}

\begin{document}
\lstset{language=Haskell} 
\setcounter{secnumdepth}{0}
\title{План лекций}
\date{}
\maketitle

\section{Введение}
Компилятор ghc, ghci, Haskell Platform.

Haskell – чисто функциональный, типизированный язык программирования.

Чистые функции.

Типы Int, Integrer, Float, Double, Bool = True | False, Char.

Арифметические операции.

Тип функции:
\begin{lstlisting}
and :: Bool -> Bool -> Bool
and False _ = False
and True x = x
\end{lstlisting}

Кортежи (a,b).
fst, snd.

Списки
\begin{lstlisting}
[a] = [] | a : [a]
[]
1:2:[]
[1,2]
[1..3] = [1,2,3]
[1,1.5..3] = [1.0,1.5,2.0,2.5,3.0]
\end{lstlisting}

Конструктор списков (list comprehensions)
\begin{lstlisting}
[x | x <- [1..3]] = [1,2,3]
[(x,y) | x <- [1,2], y <- [1,2]] = [(1,1), (1,2), (2,1), (2,2)]
[(x,y) | x <- [1..3], y <- [1..4], x == y] = [(1,1), (2,2), (3,3)]
\end{lstlisting}

\section{Базовые функции со списками}
\begin{lstlisting}
head :: [a] -> [a]
head (x:xs) = x
tail :: [a] -> [a]
tail (x:xs) = xs

(++) :: [a] -> [a] -> [a]
(++) [] ys     = ys
(++) (x:xs) ys = x : (xs ++ ys)

(x:_)  !! 0 = x
(_:xs) !! n = xs !! (n-1)

reverse :: [a] -> [a]
reverse [] = []
reverse (x:xs) = reverse xs ++ [x]

reverse l =  rev l [] where
    rev []     a = a
    rev (x:xs) a = rev xs (x:a)

take :: Int -> [a] -> [a]
take _ []     = []
take n (x:xs) | n <= 0    = []
              | otherwise = x : take (n-1) xs

drop

splitAt :: Int -> [a] -> ([a], [a])
splitAt n xs = (take n xs, drop n xs)
\end{lstlisting}

\section{Бесконечные списки}
\begin{lstlisting}
[1..]
[2,4..]
take 5 [1..]               
\end{lstlisting}

\section{Функции высших порядков}

\begin{lstlisting}
takeWhile :: (a -> Bool) -> [a] -> [a]
takeWhile _ []     = []
takeWhile p (x:xs) | p x       = x : takeWhile p xs
                   | otherwise = []

dropWhile                  
\end{lstlisting}

\section{Свёртка}

\begin{lstlisting}
sum []     = 0
sum (x:xs) = x + sum xs

product []     = 1
product (x:xs) = x * product xs

concat []       = []
concat (xs:xss) = xs ++ concat xss

foldr :: (a -> b -> b) -> b -> [a] -> b
foldr f e []     = e
foldr f e (x:xs) = f x foldr f e xs

foldl :: (b -> a -> b) -> b -> [a] -> b
foldl f e []     = e
foldl f e (x:xs) = foldl f (f e x) xs
\end{lstlisting}
\end{document}
