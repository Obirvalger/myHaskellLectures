\documentclass[a4paper,10pt]{article}
\usepackage[utf8]{inputenc}
\usepackage[russian]{babel}
\usepackage{listings}
\usepackage{fullpage}
\usepackage{hyperref}

\begin{document}
\setcounter{secnumdepth}{0}
\title{Список литературы по Haskell}
\date{}
\maketitle
\section{На русском:}
\begin{itemize}
\item {
\href{https://anton-k.github.io/ru-haskell-book/book/home.html}{"Учебник по Haskell"}, Антон Холомьёв.

Один из наиболее полных учебников по Haskell;
}

\item {
\href{https://www.ohaskell.guide/}{"О Haskell по-человечески"}, Денис Шевченко.

Книга написана доступным языком, но слишком практическое описание;
}

\item {
"14 занимательных эссе о языке Haskell и функциональном программировании"{}, Роман Душкин.

Набор рассуждений о конструкциях языка и несколько интересных примеров.
}
\end{itemize}


\section{На английском:}
\begin{itemize}
\item {
\href{http://learnyouahaskell.com/chapters}{"Learn You a Haskell for Great Good! (Изучай Haskell во имя добра!)"}.

В интересной и весёлой форме рассказывается о языке Haskell;
}

\item {
\href{http://book.realworldhaskell.org/read/}{"Real World Haskell"}.

Фундаментальная книга, но сложновата;
}

\item {
\href{https://en.wikibooks.org/wiki/Haskell}{"Haskell wikibook"}.

Довольно краткий учебник по Haskell, при этом содержит некоторые уникальные статьи.
}
\end{itemize}

\end{document}
